\documentclass[11pt,a4paper]{article}
\usepackage{geometry}  % gör texten lite bredare eftersom vanlig latex är för smal för a4

% minska storleken på rubriker och underrubriker
\usepackage{titlesec}
\titleformat{\section}
    {\normalfont\Large\bfseries}{\thesection}{1em}{}

\titleformat{\subsection}
    {\normalfont\normalsize\bfseries}{\thesubsection}{1em}{}[\vspace{-0.4em}]
\usepackage[]{parskip}

% undvik helst subsubsections

%% FÖRKLARINGAR AV KOMMANDON OCH DYL.
% \\ skapar en radbrytning
% \\[1em] skapar en radbrytning med "1em" extra utrymme mellan rader
%   (1 em är ganska stor)
% \textbf{Skapar text i fetstil}
% \begin{itemize} och \end{itemize} skapar en s.k. list-enviornment (list-miljö)
%   Inuti itemize skriver man \item för att skapa en ny punkt, annars är det bara att skriva
% \section*{Titel} och \subsection*{Understitel} skapar nya sektioner och undersektioner
%   Asterisken ser till att det inte dyker upp ett nummer bredvid namnet.
% I övrigt är det bara att skriva text där man vill ha den

\begin{document}
\pagenumbering{gobble} % räkna inte (och skriv inte heller ut) sidnumrering
    \begin{center}
        \textbf{\Large TFYY51 - Yc4} \\[0.2em]
        Mötesprotokoll för möte den 27 november i Bilbanerummet.
    \end{center}
    \vspace{1em}
    \textbf{Närvarande:} David (ordf), Alexander (sek), Mattias (lokal) \\[0.5em]
    \textbf{Frånvarande:}  \\[0.5em]  % som exempel
    \textbf{Övriga närvarande: }Gustav, Albin

    \section*{Föregående protokoll}
    \begin{itemize}
        \item Föregående möte handlade övergripande om att bana 2 inte fungerade samt att kontrollera vad som skulle göras till nästa möte.\
    \end{itemize}

    \section*{Dagordning}
    \subsection*{Tictoc}
	    Detta togs upp på mötet då tictoc har haft för högt värde men det kommer kunna lösa utan några större problem.
	\subsection*{Teknisk dokumentationen}
	    Under mötet gick vi igenom hur långt vi hade kommit i den tekniska dokummentationen.
	\subsection*{Begräsningar i vår kod}
	 	Det som fungerar för tillfället är att alla bilarna fungerar att starta med utan problem. Vi disskuterade också hurvida alla krav som förhandlades med Erik kommer vara genomförbara. För tillfället går det att justera referenstiden uppåt och nedåt. Dock är det oklart om systemet kommer följa kraven så som standardavvikelser etc. 
    \section*{Övrigt}
    

    \section*{Nästa möte}
    Nästa mötes hålls måndagen den andra december kl. 13:00. Lokalbokare kommer då vara Albin. Sekreterare till nästa möte är Mattias och ordförande till mötet är Alexander.  
\end{document}
