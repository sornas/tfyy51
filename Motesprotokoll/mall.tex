\documentclass[11pt,a4paper]{article}
\usepackage{geometry}  % gör texten lite bredare eftersom vanlig latex är för smal för a4

% minska storleken på rubriker och underrubriker
\usepackage{titlesec}
\titleformat{\section}
    {\normalfont\Large\bfseries}{\thesection}{1em}{}

\titleformat{\subsection}
    {\normalfont\normalsize\bfseries}{\thesubsection}{1em}{}[\vspace{-0.4em}]
\usepackage[]{parskip}

% undvik helst subsubsections

%% FÖRKLARINGAR AV KOMMANDON OCH DYL.
% \\ skapar en radbrytning
% \\[1em] skapar en radbrytning med "1em" extra utrymme mellan rader
%   (1 em är ganska stor)
% \textbf{Skapar text i fetstil}
% \begin{itemize} och \end{itemize} skapar en s.k. list-enviornment (list-miljö)
%   Inuti itemize skriver man \item för att skapa en ny punkt, annars är det bara att skriva
% \section*{Titel} och \subsection*{Understitel} skapar nya sektioner och undersektioner
%   Asterisken ser till att det inte dyker upp ett nummer bredvid namnet.
% I övrigt är det bara att skriva text där man vill ha den

\begin{document}
\pagenumbering{gobble} % räkna inte (och skriv inte heller ut) sidnumrering
    \begin{center}
        \textbf{\Large TFYY51 - Yc4} \\[0.2em]
        Mötesprotokoll för DATUM i SAL
    \end{center}
    \vspace{1em}
    \textbf{Närvarande:} Gustav (ordf), David (sek), Alexander (lokal), Mattias, Albin \\[0.5em]
    \textbf{Frånvarande:} Rutt och Tuck (avrättning??) \\[0.5em]  % som exempel
    \textbf{Övriga närvarande:}

    \section*{Föregående protokoll}
    \begin{itemize}
        \item Sak som hände förra mötet. \\
        Antingen kommenterade vi vad som hände förra mötet.
        \item En till sak som hände förra mötet, men större. \\
        Se dagordningen.
    \end{itemize}

    \section*{Dagordning}
    \subsection*{En sak vi vill ta upp}
    Det här är en större punkt vi vill ta upp.
    \subsection*{En annan sak}
    Det här är en annan större punkt vi vill ta upp.

    \section*{Övrigt}
    \begin{itemize}
        \item Mindre saker behöver inte ta så stor plats.
        \item Mellanstora saker kanske behöver listor i listor.
        \begin{itemize}
            \item Sådana ser ut såhär
        \end{itemize}
    \end{itemize}

    \section*{Nästa möte}
    Nästa mötes hålls blabla. Lokalbokare kommer då vara Person D.  
\end{document}