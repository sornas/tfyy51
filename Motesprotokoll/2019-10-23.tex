\documentclass[11pt,a4paper]{article}
\usepackage{geometry}  % gör texten lite bredare eftersom vanlig latex är för smal för a4

% minska storleken på rubriker och underrubriker
\usepackage{titlesec}
\titleformat{\section}
    {\normalfont\Large\bfseries}{\thesection}{1em}{}

\titleformat{\subsection}
    {\normalfont\normalsize\bfseries}{\thesubsection}{1em}{}[\vspace{-0.4em}]
\usepackage[]{parskip}

% undvik helst subsubsections

%% FÖRKLARINGAR AV KOMMANDON OCH DYL.
% \\ skapar en radbrytning
% \\[1em] skapar en radbrytning med "1em" extra utrymme mellan rader
%   (1 em är ganska stor)
% \textbf{Skapar text i fetstil}
% \begin{itemize} och \end{itemize} skapar en s.k. list-enviornment (list-miljö)
%   Inuti itemize skriver man \item för att skapa en ny punkt, annars är det bara att skriva
% \section*{Titel} och \subsection*{Understitel} skapar nya sektioner och undersektioner
%   Asterisken ser till att det inte dyker upp ett nummer bredvid namnet.
% I övrigt är det bara att skriva text där man vill ha den

\begin{document}
\pagenumbering{gobble} % räkna inte (och skriv inte heller ut) sidnumrering
    \begin{center}
        \textbf{\Large TFYY51 - Yc4} \\[0.2em]
        Mötesprotokoll för möte den 2019-10-23 i I204.
    \end{center}
    \vspace{1em}
    \textbf{Närvarande:} 
	Albin (ordf),
	Gustav (sek), 
	David (lokal), 
	Alexander.
	\\[0.5em]
	\textbf{Frånvarande:} Mattias (bortrest) \\[0.5em]
    \textbf{Övriga närvarande:}

    \section*{Föregående protokoll}
	Inget föregående protokoll att gå igenom.

    \section*{Dagordning}
    \subsection*{Arbetsdisponering under tenta-/omtenta-p}
	Nästa leverans är BP4b under projektvecka 7 (veckan med start 2019-11-11).
	Tills dess vill vi ha implementerat alla de större funktionerna som
	specifierades i designspecifikationen. Vi beslutade att Alexander och David
	redan imorgon (2019-10-24) ska göra mätningar av banan, dels längd på olika
	undersegment och dels mätning av spänning i dessa.

	Vi gick även igenom det som fanns kvar efter leveransen av BP4a. För att
	fördela arbetet delade vi ut issues som fanns på Gitlab (och skrev in några
	nya som vi kom på under mötets gång).

	\subsection*{Beslut om struktur för gamla värden på position och hastighet}

	Under leveransen av designspecifikationen lyfte Erik fram möjligheten att
	spara hastigheten i olika delar av banan under tidigare varv som en extra
	datapunkt när den nuvarande hastigheten ska bestämmas. Vi beslutade att
	implementera detta genom att utgå från kartan som Alexander och David bygger
	upp för att få de olika segmenten, och sedan spara värdena i en
	tre-dimensionell matris. Matrisen har formated ((antal~varv) $\times$ (antal~
	undersegment) $\times$ 2) där $z$-ledet beskriver hastigheten och varvtiden
	vid ett visst segment vid ett tidigare varv. Fördelen är att systemet vid en
	viss tidpunkt kan applicera statistiska funktioner på de tidigare
	mätpunkterna för ett visst segment som ett hjälpmedel vid nuvarande
	beräkningar. Dessa mätvärden kan med lätthet utökas för att spara mer
	information vid olika segment.

	För att underlätta prestandamässigt kan matrisen vid start göras så stor den
	behöver vara givet att användaren matar in hur många varv en viss bil ska
	göra.

    \section*{Övrigt}
    \begin{itemize}
		\item \textbf{Beslut: } Under utveckling ska vi spara så mycket
			information som möjligt för att bygga statistik över tid.
		\item Vi gick igenom hur Gitlab automatiskt kan stänga issues med
				magiska keywords (ex. "implements \#20" eller "fixes \#5").
		\item Vi beslutade att utveckling mellan leveranser i huvudsak sker
				med en ny branch per feature som sedan mergas till master. I
				praktiken använder vi oss därför från och med nu av Merge 
				Requests (i den mån det är möjligt).
    \end{itemize}

    \section*{Nästa möte}
    Nästa mötes hålls 2019-10-29. Lokalbokare kommer då vara Alexander.
	Förhoppnigen är att alla medlemmar kan närvara. 
\end{document}
