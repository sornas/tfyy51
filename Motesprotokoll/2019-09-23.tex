\documentclass[11pt,a4paper]{article}
\usepackage{geometry}  % gör texten lite bredare eftersom vanlig latex är för smal för a4
\usepackage{url}

% minska storleken på rubriker och underrubriker
\usepackage{titlesec}
\titleformat{\section}
    {\normalfont\Large\bfseries}{\thesection}{1em}{}

\titleformat{\subsection}
    {\normalfont\normalsize\bfseries}{\thesubsection}{1em}{}[\vspace{-0.4em}]
\usepackage[]{parskip}

% undvik helst subsubsections

%% FÖRKLARINGAR AV KOMMANDON OCH DYL.
% \\ skapar en radbrytning
% \\[1em] skapar en radbrytning med "1em" extra utrymme mellan rader
%   (1 em är ganska stor)
% \textbf{Skapar text i fetstil}
% \begin{itemize} och \end{itemize} skapar en s.k. list-enviornment (list-miljö)
%   Inuti itemize skriver man \item för att skapa en ny punkt, annars är det bara att skriva
% \section*{Titel} och \subsection*{Understitel} skapar nya sektioner och undersektioner
%   Asterisken ser till att det inte dyker upp ett nummer bredvid namnet.
% I övrigt är det bara att skriva text där man vill ha den

\begin{document}
\pagenumbering{gobble} % räkna inte (och skriv inte heller ut) sidnumrering
    \begin{center}
        \textbf{\Large TFYY51 - Yc4} \\[0.2em]
        Mötesprotokoll för 23 September 2019
    \end{center}
    \vspace{1em}
    \textbf{Närvarande:} Gustav (ordf), David (sek), Alexander (lokal), Mattias, Albin \\[0.5em]
    \textbf{Frånvarande:} \\[0.5em]  % som exempel
    \textbf{Övriga närvarande:}

    \section*{Föregående protokoll}
    \vspace{1em}
    
    \section*{Dagordning}
    \subsection*{Stämma av projektplan}
    Inte direkt några invändningar. Projektplanen ser bra ut helt enkelt
    \subsection*{Gör mailgrupp}
    Mattias skapar en gemensam mailgrupp; \url{team_YC4@liu.onmicrosoft.com}, så att vi kan ha mailkontakt med varandra på ett smidigt och funktionellt sätt.
    Dessutom kan Eriks information nå ut till hela gruppen skulle han ha något viktigt att säga.
    \subsection*{Mer detaljrik tidsplan}
    Vi räknar ut att vi vill snitta på 10 timmar/pers/vecka. Färre timmar, ungefär hälften av snittet, kan läggas ned på arbetet de veckor som kräver förberedelse inför diverse
    prov eller redovisningar
    \subsection*{Gör mötesprotkollmall}
    Mallen är strukturerad på det sätt vi vill ha det. Mallen skrivs in i \LaTeX\ och läggs upp i gitLab
    \subsection*{Formell gruppbestämmelse}
    Utökat gruppkontrakt, med bestämmelse om struktur och ordning i gruppen. Bland annat mötesordning och rullians på roller inför möten

    \section*{Övrigt}
    \vspace{1em}

    \section*{Nästa möte}
    Nästa mötes hålls Tisdag 1 Oktober 2019 13:15. Lokalbokare kommer då vara Mattias.  
\end{document}