\documentclass[10pt,swedish,oneside]{lips-no_customer}

\usepackage{multirow}
\usepackage[round]{natbib}\bibliographystyle{plainnat}

\title{Gruppbestämmelser}
\author{Yc4}
\date{2019-10-04}
\version{1.0}

\reviewed{}{}
\approved{Gruppmedlemmarna}{2019-10-04}

\projecttitle{Styrning och optimering av bilbana}

\groupname{Yc4}
\groupemail{team_yc4@liuonline.onmicrosoft.com}
\groupwww{https://www.fs.isy.liu.se/Edu/Courses/TFYY51/}

\coursecode{TFYY51}
\coursename{Ingenjörsprojekt}

\orderer{Erik Frisk, Linköpings universitet}
\ordererphone{+46 (0)13-285714}
\ordereremail{erik.frisk@liu.se}

% \customer{Kund, Företag X}
% \customerphone{+46 xxxxxx}
% \customeremail{customer@companyx.com}

\courseresponsible{Urban Forsberg}
\courseresponsiblephone{+46 (0)13-281350}
\courseresponsibleemail{urban.forsberg@liu.se}

\supervisor{Viktor Leek}
\supervisorphone{+46 (0)13-284493}
\supervisoremail{viktor.leek@liu.se}

\cfoot{\thepage}

\begin{document}
\maketitle

\cleardoublepage
\tableofcontents
\cleardoublepage

\section*{Dokumenthistorik}
    \begin{tabular}{p{.06\textwidth}|p{.1\textwidth}|p{.45\textwidth}|p{.13\textwidth}|p{.12\textwidth}}
        \multicolumn{1}{c}{\bfseries Version} &
        \multicolumn{1}{|c}{\bfseries Datum} &
        \multicolumn{1}{|c}{\bfseries Utförda förändringar} &
        \multicolumn{1}{|c}{\bfseries Utförda av} &
        \multicolumn{1}{|c}{\bfseries Granskad}\\
        \hline
        \hline
        0.1 & 2019-09-29 & Första utkast & Gustav & \\\hline
	    	1.0 & 2019-10-04 & Antagen av gruppen & -- \\\hline        
    \end{tabular}

    \cleardoublepage
    \pagenumbering{arabic}\cfoot{\thepage}

    \section{Möten}
    Datum och tid för möten planeras i första hand av Mattias. Mattias är även ansvarig för dagordingen. Om projektmedlemmar vill lägga till något på dagordingen till nästa möte ska de i första hand skicka ett privat meddelande till Mattias i god tid innan mötet. Om projektmedlemmar vill ta upp något mindre räcker det att ta upp det under punkten ``Övrigt'' under mötet.

    Mötesprotokoll skrivs av sekreteraren. Mall för mötesprotkoll finns på Gitlab. I den mån det är möjligt ska sekreteraren skriva mötesprotokollet enligt mallen och typsätta med \LaTeX.

    Till varje möte ska ordförande, sekreterare och lokalbokare utses. Dessa positioner roteras i följande ordning.
    \begin{enumerate}
        \item Gustav
        \item David
        \item Alexander
        \item Mattias
        \item Albin
    \end{enumerate}
    Ordföranden börjar på nummer 1, sekreteraren på nummer 2 och lokalbokaren på nummer 3. Platserna roteras sedan i den ordning de står i listan ovan. Om någon ansvarig inte kan närvara vid ett möte tar nästa lediga person över rollen som ``ställföreträdare'' (noteras med ``stf.'' i protokollet). Om 1, 2, 3 skulle ansvara för ett möte men 3 är borta kommer då 1, 2, 4, ansvara för mötet och 2, 3, 4 för nästkommande möte.

    \section{Gruppkontrakt}
        Gruppmedlemmarna har var och en gått igenom gruppkontraktet från LIPS-mallarna och fyllt i hur de personligen känner inför elva påståenden på en skala mellan 1 och 4. Nedan följer en sammanställning om hur mycket poäng varje punkt fick i snitt, alltså ungefär var gruppens prioriteringar bör ligga. \vspace{1em}

        \begin{tabular}{|l|l|} \hline
            \bfseries Påstående & \bfseries Poäng \\\hline
            Arbetsuppgifterna ska fördelas livärdigt mellan gruppmedlemmarna & \multirow{2}{3em}{3.5} \\ så att arbetsinsatsen blir ungefär lika stor. &  \\\hline
            Alla deltagare i gruppen ska delta vid de tillfällen som gruppen kommit överens om. & 3.25 \\\hline
            Vår grupps ambitionsivå är att arbetet ska leda till att det framtagna resultatet i projektet & \multirow{2}{3em}{3.25} \\ blir det bästa tänkbara. & \\\hline
            Alla deltagare i gruppen ska komma väl förberedda till sammankomsten. & 3 \\\hline
            Det är viktigt med en ``sammordnare'' i gruppen. & 2.75 \\\hline
            Samarbetet i gruppen måste när som helst kunna diskuteras öppet, & \multirow{2}{3em}{2.75} \\ även om det innebär obehag för någon. & \\\hline
            Deltagarna i gruppen ska gemensamt göra upp ordningsregler för gruppen & \multirow{2}{3em}{2.75} \\ om tider, närvaro, förberedelser, etc. & \\\hline
            Den som inte bidrar aktivt ska inte heller dra nytta av gruppens gemensamma arbete. & 2.5 \\\hline
            Varje träff avslutas med en utvärdering, där var och en belyser & \multirow{2}{3em}{1.75} \\ hur arbetet i gruppen fungerat. & \\\hline
            När vi arbetar i gruppen ska vi hålla oss till fakta och undvika prat om känslor & \multirow{2}{3em}{1.25} \\ och personliga erfarenheter. &\\\hline
        \end{tabular}
    \section{Övrigt}
        Vid större milstolpar (del-leveranser och slut-leveranser) ska gruppen äta en tårta. Tårtan införskaffas gemensamt.
\end{document}
