\documentclass[10pt,oneside,swedish]{lips}

%\usepackage[square]{natbib}\bibliographystyle{plainnat}\setcitestyle{numbers}
\usepackage[round]{natbib}\bibliographystyle{plainnat}
\usepackage{parskip}

\RequirePackage[yyyymmdd]{datetime}
\renewcommand{\dateseparator}{-}

% Configure the document
\title{Efterstudie}
\author{Yc4}
\date{\today}
\version{0.1	}

\reviewed{}{}
\approved{}{}

\projecttitle{Styrning och optimering av bilbana}

\groupname{Yc4}
\groupemail{team\_yc4@liuonline.onmicrosoft.com}
\groupwww{https://www.fs.isy.liu.se/Edu/Courses/TFYY51/}

\coursecode{TFYY51}
\coursename{Ingenjörsprojekt}

\orderer{Erik Frisk, Linköpings universitet}
\ordererphone{+46(0)13-285714}
\ordereremail{erik.frisk@liu.se}

%\customer{Kund, Företag X}
%\customerphone{+46 xxxxxx}
%\customeremail{erik.frisk@liu.se}

\supervisor{Viktor Leek, Linköpings universitet}
\supervisorphone{+46(0)13-284493}
\supervisoremail{viktor.leek@liu.se}

\courseresponsible{Urban Forsberg, Linköpings universitet}
\courseresponsiblephone{+46(0)13-281350}
\courseresponsibleemail{urban.forsberg@liu.se}

\smalllogo{logo} % Page header logo, filename
\biglogo{logo} % Front page logo, filename

\cfoot{\thepage}
\begin{document}
\maketitle

\cleardoublepage
\makeprojectid

\begin{center}
  \Large Projektdeltagare
\end{center}
\begin{center}
  \begin{tabular}{|l|l|l|} \hline
    \textbf{Namn} & \textbf{Ansvar} & \textbf{Kontaktinformation }\\\hline

    Mattias Uvesten & Projektledare (PL) & 0768697559\\
    && \url{matvu053@student.liu.se} \\\hline

    Gustav Sörnäs & Dokument, display (DOK, DSP) & 0703279113\\
    && \url{gusso230@student.liu.se} \\\hline

    Alexander Tuneskog & Hastighet (SPD) & 0725559873 \\
    && \url{aletu130@student.liu.se} \\\hline

    David Thorén & Tester, gemensam målgång (TST, GML) & 0721838605 \\
    && \url{davth346@student.liu.se} \\\hline

    Albin Wahlén & Kalibrering, positionering (KLB, POS) & 0762016054 \\
    && \url{albwa833@student.liu.se} \\\hline
  \end{tabular}
\end{center}

\cleardoublepage
\tableofcontents

\cleardoublepage
\section*{Dokumenthistorik}
\begin{tabular}{|p{.06\textwidth}|p{.1\textwidth}|p{.45\textwidth}|p{.13\textwidth}|p{.13\textwidth}|} 
  \hline
  \multicolumn{1}{|c}{\bfseries Version} &
  \multicolumn{1}{|c}{\bfseries Datum} &
  \multicolumn{1}{|c}{\bfseries Utförda förändringar} &
  \multicolumn{1}{|c}{\bfseries Utförda av} &
  \multicolumn{1}{|c|}{\bfseries Granskad}\\
  \hline \hline
  &&&& \\
  \hline
\end{tabular}

\cleardoublepage
\pagenumbering{arabic}\cfoot{\thepage}

\section{Tidsåtgång}

Nedan redovisas hur gruppens tidsfördelning under projektet sett ut.

\subsection{Arbetsfördelning}

Gruppen anser att arbetsfördelningen i stort har fungerat bra, men de flesta
gruppmedlemmar lade ner mer tid än planerat (som mest upp emot 50 timmar mer än
de 120 inplanerade).

Gruppmedlemmarna upplever i efterhand att de spenderade för mycket tid på en
lösning som i slutändan inte visade sig vara tillräcklig för att lösa problemet.
Detta upptäckes för sent i arbetets gång vilket gjorde att orimligt mycket tid
behövde läggas på att få en färdig slutprodukt i slutet av arbetet. 

\subsection{Jämförelse med planerad tid}

Nedan återfinns en tabell över planerad tid och använd tid för vardera fas i
utvecklingen. 

\begin{tabular}{|l|l|l|} \hline
	\textbf{Fas} & \textbf{Planerad tid i timmar} & \textbf{Använd tid i timmar} \\\hline
	Före & 22 h & 120 h \\\hline
	Under & 358 h & 450 h \\\hline
	Efter & 150 h & 170 h \\\hline
	\textbf{Summa} & \textbf{600 h (varav 70 h buffer)} & \textbf{740 h} \\\hline
\end{tabular}

Gruppen anser att den använda andelen tid per fas ser bra ut. För att få summan
till 600 timmar istället för nuvarande 740 hade gruppen önskat att efter-fasen
hade kortats ner till 120 timmar, samt att under-fasen hade kortats ner till 360
timmar för att få en 20/60/20-fördelning mellan de olika faserna.

\section{Analys av arbete och problem}

I den här delen presenteras hur gruppen arbetat med projektet och hur
medlemmarna upplevt arbetsprocessen.

\subsection{Faserna} \label{sec:faser}

Arbetet var indelat i tre faser enligt LIPS-modellen. Dessa redovisas och
analyseras nedan.

\subsubsection{Före-fasen}

Under före-fasen planerades projektet och hur systemet skulle se ut. Före-fasen
gjorde att gruppen kände att de hade något att utgå ifrån när utvecklingen väl
började. Tyvärr kände gruppmedlemmarna under själva utförandet av före-fasen att
det var svårt att motivera framställningen av dokument som upplevdes som
onödiga. Allt eftersom visade det sig dock att dessa dokument hjälpte gruppen
när systemet skulle utvecklas.

Gruppen önskar att de i före-fasen lagt mer tid på att ta reda på teorin bakom
styrning av liknande system.

\subsubsection{Under-fasen}

Under utvecklingen av systemet fokuserade gruppen på att få alla gruppmedlemmar
att börja arbeta med banan och skriva egen kod. Eftersom samtliga gruppmedlemmar
utom en saknade erfarenhet av programmering sedan innan var det gruppens önskan
att alla skulle ha möjligheten att hjälpa till i utvecklingen. Tyvärr gjorde en
enskild gruppmedlem majoriteten av det tidiga arbetet vilket gjorde att övriga
gruppmedlemmar hade svårt att hjälpa till med utvecklingen. En hjälpande faktor
var dock arbetsmetodiken; gruppen använde sig tidigt av git och Gitlab vilket
gjorde att samtliga gruppmedlemmar kunde följa arbetet och allt eftersom arbetet
fortsatte göra sina egna tillägg till systemet.

För att motverka att en enskild person gör en stor del av det tidiga
arbetet rekommenderar gruppen att alla gruppmedlemmar så fort som möjligt gör
egna implementationer av problemen och tillsammans går igenom dem.

\subsubsection{Efter-fasen}

Arbetet under efter-fasen präglades av effektivitet. Gruppen hade missförstått
deadline för teknisk dokumentation och behövde därför göra mer arbete än
planerat under kort tid. För att motverka denna missuppfattning borde gruppen
under före-fasen sammanställt samtliga deadlines och andra viktiga datum samt
internt bestämt egna deadlines i god tid innan respektive leverans och
dokumentinlämning.

\subsection{Samarbete}

Ett av huvudsyftena med kursen var att ``aktivt medverka i en fungerade
projektgrupp''. \footnote{\url{https://liu.se/studieinfo/kurs/tfyy51/ht-2018}}
Nedan reflekterar gruppen hur samarbetet fungerat.

\subsubsection{Ansvarsområden}

Ansvarsområdena som de specifierades i projektplanen har delvis följts.
Rollerna för projektledare och dokumentansvarig har varit nödvändiga och har
således funnits genom hela projektet. Ansvarsområdena för utveckling föll
däremot samman under utvecklingen. Gruppen gissar att detta berodde på systemets
relativt lilla storlek och att det i ett större projekt är tydligare var ett
ansvarsområde tar slut och ett annat börjar. Det ansvarsområde inom utvecklingen
som fortfarande efterföljdes vid projektets slut var displayen, troligen på
grund av den tydliga avgränsningen mellan den och bilkörningen.

\subsubsection{Kommunikation}

Gruppen upplever att kommunikationen har fungerat bra. Under arbetets gång har
en gruppchatt använts för att samla all kommunikation på samma ställe (förutom
när hela gruppen varit samlad). Några veckor in i arbetet skapade gruppen också
en gemensam kalender vilket underlättade planeringen av gemensamma möten och
arbetstillfällen.

\subsubsection{Beslut}

Alla meningsskiljaktigheter i gruppen kunde lösas med gemensam diskussion utan
att projektledaren (som i någon mening får anses ha sista ordet) behövde ta ett
exekutivt beslut.

\subsection{Projektmodellen}

Dokument framtogs i enlighet med projektmodellen. Gruppen anser särskilt att
efterstudien har varit mycket givande som ett verktyg för reflektion av
samarbete och projektarbete i helhet. Gruppen upplevde projektmodellen som lite
för stor jämfört med projektet men accepterar samtidigt att storleken är rimlig
med avseende på kursens syfte.

\subsection{Relationen med beställaren}

Relationen med beställaren (Erik Frisk) har fungerat bra.

\subsection{Relationen med handledaren}

Relationen med handledaren (Viktor Leek) har fungerat synnerligen bra.
Handledaren har kommit med hård, relevant och rättvis kritik mot inlämnade
dokument och hjälpt gruppen nå resultat de inte själva trodde de kunde nå.

\subsection{Tekniska framgångar och problem}

Vissa delar av systemet fungerande över förväntan medan andra inte nådde upp
till kraven som ställdes. Anledningen till detta diskuteras ovan
(\ref{sec:faser}). Runt projektvecka 8 havererade själva bilbanan vilket gjorde
att utvecklingen temporärt stannade upp. På grund av de tekniska problemen gick
beställaren med på att omförhandla några av kraven.

\section{Måluppfyllelse}

För hela projektet fanns dels mål med kursen och dels mål med själva projektet.

\subsection{Mål med kursen}

Gruppmedlemmarna har fått erfarenhet inom grupparbete helt olikt grupparbeten de
var vana vid från gymnasiestudier. De har också förstått det viktiga i att dela
upp problem mellan personer och hur man i praktiken bygger ett system
tillsammans.

\subsection{Mål med projektet}

Själva presentationen gick bra och stämningen var god. De tekniska resultaten
var å andra sidan under förväntan, även efter kraven hade sänkts vid två
tillfällen. Dessa redovisas i fullo i den tekniska dokumentationen.

\subsection{Studiesituationen}

Medlemmarnas studiesituation har inte påverkat projektarbetet nämnvärt. Gruppen
anser snarare att motsatsen har skett; vid vissa tillfällen har projektet gått
ut över studiesituationen i övrigt.

\section{Sammanfattning}

Gruppen anser att kursen har varit mycket givande. Tyvärr nådde själva systemet
inte upp till de krav som ställdes, vilket gruppen självklart upplevde som
tråkigt.

\subsection{De tre viktigaste erfarenheterna}

\begin{enumerate}
	\item Samarbete i grupp.
	\item Hur man designar och utvecklar ett system för att lösa ett komplext
		problem.
	\item Rimlig(-are) tidsuppskattning.
\end{enumerate}

\subsection{Goda råd till någon som utför ett liknande projekt}

\begin{itemize}
	\item Kom igång alla så fort som möjligt! Lämna inte någon bakom.
	\item Utnyttja varandras tidigare erfarenheter.
	\item Använd git (ordentligt).
	\item Utnyttja handledaren.
	\item Kolla upp teorin bakom reglering tidigt.
\end{itemize}
\appendix
\section{Tidsåtgång per aktivitet}

\begin{tabular}{|l|l|l|} \hline
	\textbf{Moment}                      & \textbf{Planerat}    & \textbf{Faktiskt}    \\\hline
Display                     &          60 &          20 \\\hline
\emph{Körning}                    &         224 &         253 \\\hline
  \ \emph{Variabel hastighet}        &         150 &           -- \\\hline
    \ \ Utifrån del av banan    &          60 &           -- \\\hline
    \ \ Utifrån körningsegenskap&          60 &         -- \\\hline
    \ \ Utifrån varvtid         &          30 &           -- \\\hline
  \ Gemensam målgång          &          24 &           -- \\\hline
  \ Missade givare            &          10 &          19 \\\hline
  \ Avåkning                  &          10 &           -- \\\hline
  \ Testning/felsökning       &          30 &          22 \\\hline
\emph{Dokument}                    &         114 &           119 \\\hline
  \ Projektplan och tidsplan  &           8 &          15 \\\hline
  \ Mötesprotokoll            &           8 &           1 \\\hline
  \ Beställarkontakt          &           8 &           5 \\\hline
  \ Teknisk dokumentation     &          80 &          88 \\\hline
  \ Dokumentansvarigsarbete   &          10 &          10 \\\hline
Övriga dokument             &           -- &          76 \\\hline
\emph{Övrigt}                      &         132 &           220 \\\hline
  \ Presentation              &          20 &          83 \\\hline
  \ Bibiloteksuppgift         &          40 &           9 \\\hline
  \ Möten                     &          58 &         103 \\\hline
  \ Utbildning                &          14 &          25 \\\hline
Övriga aktiviteter          &           -- &           8 \\\hline
\textbf{SUMMA}                       &         530 &         737 \\\hline
	
	
\end{tabular}

\end{document}


