För att optimera bilarnas körning och varvtid kommer bilarna behöva kalibreras.
Enligt kravsspecifikationens punkt 22 får det inte köras fler än 5
kalibreringsvarv innan kraven 20 och 21 måste vara uppfyllda.

Kalibreringsvarven behöver ta hänsyn till tre moment. Det första är att
identifiera vilken konstant en bil behöver, det andra är att optimera för
gemensam varvtid och det tredje är att optimera den valda varvtiden.

För att identifiera en bilkonstant börjar programmet med att välja den lägsta
spänningen som är möjlig vid starten. Efter första givaren är det möjligt att
räkna ut hastigheten samt ta fram en begynnelsekonstant för första segmentet
som kan användas för uppskattning av position. Det är också möjligt att se om
bilen ligger efter eller kör enligt tid genom att jämföra med tidigare mätdata.
Om bilen ligger efter kommer spänningen ökas succesivt tills bilen har nått den
önskade tiden mellan segmenten för att uppnå önskad varvtid. När bilen ligger i
fas kan den konstant som krävs för den nuvarande bilen räknas ut.

Under kalibreringsvarven är det också viktigt att anpassa båda bilarnas varvtid
mot varandra. Om båda bilarna har en varvtid som är långsammare än den
förväntade hastigheten ska programmet optimera för en gemensam varvtid före den
valda varvtiden. Med funktioner ska programmet också kunna optimera för både
gemensam och vald varvtid.

