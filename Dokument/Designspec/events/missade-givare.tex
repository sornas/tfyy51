Enligt krav 3 i kravspecifikationen ska programmet kunna hantera missade givare
och fortsätta köra som normalt. Med metoden som används blir detta inte ett
problem. Givarna kommer endast att användas för att justera programmets
uppfattning om bilarnas position, medan själva positioneringen räknas ut av
systemet. Om programmet detekterar att en givare inte passeras överhuvudtaget
när det var förväntat fortsätter systemet köra bilen enligt beräkningar på vart
bilen borde befinna sig tills nästa givare. Vid passering av nästa givare
kommer systemet jämföra den uträknade tiden och den faktiska tiden som bilen
passerade på. Systemet jämför sedan om tiderna är rimliga och avgör därefter om
den tidigare tidsfördröjningen var en missad givare eller något annat.

