För att optimera bilarnas körning och varvtid så kommer bilarna behöva
kalibreras. Enligt kravsspecifikation punkt 22 får det inte genomföras fler än
5 kalibreringsvarv för att uppfylla kraven till punkt 20 och 21.

Kalibreringsvarven kommer att behöva ta hänsyn till tre moment. Det första momentet är att
identifiera vilken konstant en bil behöver, det andra momentet är att optimera för
gemensam varvtid och det tredje är att optimera den valda varvtiden.


För att identifiera en bilkonstant börjar programmet med att välja den lägsta spänningen som är möjlig vid starten (dvs utgå från en kall bana samt den bil som kräver mest spänning för att få den till att rulla). Efter första givaren är det möjligt att räkna ut hastigheten samt ta fram en begynnelsekonstant för första segmentet som kan användas för uppskattning av position. Det är också möjligt att se om bilen ligger efter eller kör enligt tid (dvs programmet utgår från uppmätta snitt-tider från tidigare mätdata när bilarna har hållt sig enligt tidsplanen). Om bilen ligger efter kommer spänningen öka succesivt tills bilen har nått den önskade tiden mellan segmenten för att uppnå önskad varvtid. När bilen ligger i fas kan en konstant räknas ut som bilen behöver. 

Under
kalibreringsvarven är det också viktigt att anpassa båda bilarnas varvtid med
varandra. Om båda bilarna har en varvtid som är långsammare än den förväntade
hastigheten så ska programmet optimera för en gemensam varvtid före den valda
varvtiden. Med funktioner ska programmet också kunna optimera för både
gemensam och vald varvtid.

