För att optimera bilarnas körning och varvtid så kommer bilarna behöva
kalibreras. Enligt kravsspecifikation punkt 22 får det inte genomföras fler än
5 kalibreringsvarv för att uppfylla kraven till punkt 20 och 21.

Kalibreringsvarven kommer att behöva ta hänsyn till tre moment. Den första att
identifiera vilken konstant en bil behöver, den andra för att optimera för
gemensam varvtid och den tredje för att optimera den valda varvtiden. Vid
kalibreringsvarven kommer den mesta kalibreringen göras för att identifiera
vilken konstant en bil har. Med indata kommer programmet kunna göra beräkningar
om vilka konstanter som ska användas till vardera bilars körning. Under
kalibreringsvarven är det också viktigt att anpassa båda bilarnas varvtid med
varandra. Om båda bilarna har en varvtid som är långsammare än den förväntade
hastigheten så ska programmet optimera för en gemensam varvtid före den valda
varvtiden. Med funktioner ska programmet också kunna optimera för både
gemensam och vald varvtid.

