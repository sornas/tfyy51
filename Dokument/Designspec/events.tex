\section{Hantering av händelser}
Under körning kommer det uppstå oförutsägbara händelser som inte är ett fel i programmet och detta ska programmet kunna hantera. De händelser programmet ska kunna ta hänsyn till är start, missade givare, avåkning och manuell åkning.

\subsection{Start} Innan målgivaren hittar vi ett läge där alla bilar är körbara och inte fastnar. Till första givaren behåller bilen den spänning den behövde för att börja rulla och vi kan därefter veta hur lång tid det tagit mellan start och och första givaren. Med det kan vi räkna ut vilket vilken konstant (k) som bilen behöver. 

\subsection{Avåkning} Programmet ska detektera att en bil har åkt av banan inom 10 sekunder. Detta
ska göras genom att felvarna, avbryta programmet och skriva ut detta på
displayen om programmet inte registrerar en ny givare inom dessa tio sekunder.


\subsection{Missade givare} Enligt krav 3 i kravspecifikationen ska programmet kunna hantera missade givare
och fortsätta köra som normalt. Med den metod som kommer användas blir detta inte ett problem.
Givarna kommer endast att användas för att justera programmets uppfattnting om bilarnas position, 
själva positionen ska beräknas på annat sätt.  Programmet ska detektera detta
fel genom att  se om en givare passeras när förväntat. Om den inte gör det och nästa
passering av givare sker när förväntat kommer programmet
identifiera den tidigare tidsfördröjningen som en missad givare.


Enligt krav 3 i kravspecifikationen ska programmet kunna hantera missade givare
och fortsätta köra som normalt. Med den metod som kommer användas blir detta inte ett problem.
Givarna kommer endast att användas för att justera programmets uppfattnting om bilarnas position, 
själva positionen ska beräknas på annat sätt.  Programmet ska detektera detta
fel genom att  se om en givare passeras när förväntat. Om den inte gör det och nästa
passering av givare sker när förväntat kommer programmet
identifiera den tidigare tidsfördröjningen som en missad givare.

\subsection{Missade givare}

Enligt krav 3 i kravspecifikationen ska programmet kunna hantera missade givare och fortsätta köra som normalt. Som vi har tänkt att använda oss av givarna så ska inte bilarna ändra sin körning vid ett sådant utfall. Då vi tänkt att använda givarna som en referens och inte justering av bilarnas körning så kommer en missad givare ge ett fel på referens. Programmet ska detektera detta fel genom att med dess interna lagring av data se om det nästa förväntade tidspassering stämmer överrens med nästa givare detektering. Om nästa förväntade tid stämmer överrens med en givares detektering kommer programmet identifiera den tidigare tidsfördröjningen som en missad givare.

Enligt kravspecifikationens punkt 12 ska det vara möjligt att välja om en bana
ska köras manuellt eller autonomt. Det ska alltså gå att köra ena banan
manuellt medan den andra styrs av programmet. Detta styrs via displayen vid
uppstart.


\subsection{Manuell körning}
Enligt kravspecifikationens punkt 12 ska de två olika banorna delas upp så att ena banan styrs autonomt och den andra manuellt. 
Den manuella delen ska bli hjälpt av programmet för att underläta körning vid händelse av driftfall samt uppvärmning av banan.
Detta ska uppnås genom att jämföra vilken hastighet bilen erhåller i ett visst segment styrt av vilken spänningspåläggning som verkar på bilen.
Sedan ska programmet  jämföra hastigheten med en tidigare föreslagen hastighet och sedan modifiera en konstant för att matcha det önskade värdet.

Programmet ska detektera att en bil har åkt av banan inom 10 sekunder. Detta ska göras genom att felvarna, avbryta programmet och skriva ut detta på displayen om programmet inte registrerar en ny givare inom dessa tio sekunder.
