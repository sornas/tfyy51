\section{Hantering av händelser}

\subsection{Start} För att systemet ska hitta en spänning som gör att bilen börjar rulla men inte
åker av banan i första kurvan ökas spänningen i början lite i taget tills bilen
passerar den första givaren. Efter det tar det vanliga systemet vid och
kontrollerar bilens förväntade position mot dess uträknade position.



\subsection{Avåkning} Systemet ska detektera att en bil har åkt av banan inom 10 sekunder. Om systemet
inte får en ny givarsignal inom tio sekunder från den senaste givaren antas bilen ha
fastnat eller åkt av banan och programmet pausas tills bilen är på banan igen
och användaren trycker på "fortsätt" på displayen. Med givarnas förväntade tidspassering så ska programmet kunna se att en bil inte passerat en givare vid den förväntade tidspasseringen. Har bilen inte passerat någon av givarna inom fem sekunder så kan programmet öka spänningsnivån gradvis som vid start för att kontrollera om bilen har stannat. Efter ytterligare 5 sekunder så ska programmet pausas och displayen skriva ut att bilen har åkt av banan.



\subsection{Missade givare} Enligt krav 3 i kravspecifikationen ska programmet kunna hantera missade givare
och fortsätta köra som normalt. Med metoden som används blir detta inte ett
problem. Givarna kommer endast att användas för att justera programmets
uppfattning om bilarnas position, medan själva positioneringen räknas ut av
systemet. Om programmet detekterar att en givare inte passeras överhuvudtaget när det var förväntat fortsätter systemet att köra bilen enligt beräkningar på vart bilen borde befinna sig tills nästa givare. Vid passering av nästa givare kommer systemet att jämföra om den uträknade tiden och den faktiska tiden som bilen passerade på. Där systemet jämför om tiderna är rimliga och därefter avgör om den tidigare tidsfördröjningen var en missad givare.



Enligt krav 3 i kravspecifikationen ska programmet kunna hantera missade givare
och fortsätta köra som normalt. Med metoden som används blir detta inte ett
problem. Givarna kommer endast att användas för att justera programmets
uppfattning om bilarnas position, medan själva positioneringen räknas ut av
systemet. Om programmet detekterar att en givare inte passeras överhuvudtaget när det var förväntat fortsätter systemet att köra bilen enligt beräkningar på vart bilen borde befinna sig tills nästa givare. Vid passering av nästa givare kommer systemet att jämföra om den uträknade tiden och den faktiska tiden som bilen passerade på. Där systemet jämför om tiderna är rimliga och därefter avgör om den tidigare tidsfördröjningen var en missad givare.


\subsection{Missade givare}

Enligt krav 3 i kravspecifikationen ska programmet kunna hantera missade givare och fortsätta köra som normalt. Som vi har tänkt att använda oss av givarna så ska inte bilarna ändra sin körning vid ett sådant utfall. Då vi tänkt att använda givarna som en referens och inte justering av bilarnas körning så kommer en missad givare ge ett fel på referens. Programmet ska detektera detta fel genom att med dess interna lagring av data se om det nästa förväntade tidspassering stämmer överrens med nästa givare detektering. Om nästa förväntade tid stämmer överrens med en givares detektering kommer programmet identifiera den tidigare tidsfördröjningen som en missad givare.

Enligt kravspecifikationens punkt 12 ska det vara möjligt att välja om en bana ska köras manuellt eller autonomt. Det ska alltså gå att köra ena banan manuellt medan den andra styrs av programmet.
\subsection{Manuell körning}
Enligt kravspecifikationens punkt 12 ska de två olika banorna delas upp så att ena banan styrs autonomt och den andra manuellt. 
Den manuella delen ska bli hjälpt av programmet för att underläta körning vid händelse av driftfall samt uppvärmning av banan.
Detta ska uppnås genom att jämföra vilken hastighet bilen erhåller i ett visst segment styrt av vilken spänningspåläggning som verkar på bilen.
Sedan ska programmet  jämföra hastigheten med en tidigare föreslagen hastighet och sedan modifiera en konstant för att matcha det önskade värdet.

Programmet ska detektera att en bil har åkt av banan inom 10 sekunder. Detta ska göras genom att felvarna, avbryta programmet och skriva ut detta på displayen om programmet inte registrerar en ny givare inom dessa tio sekunder.
