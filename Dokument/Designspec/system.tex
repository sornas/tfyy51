\section{Delsystem}

Systemet är indelat i två olika delsystem. Dessa system körs
sekvensiellt, alltså det ena efter det andra. Varje sekund körs de två delsystemen 10 gånger. Det första systemet kontrollerar
själva bilkörningen medan det andra systemet kontrollerar displayen. Se
figur~\ref{fig:system_diagram} för ett processchema.

\begin{figure}
  \centering
  \includegraphics[width=\linewidth,height=0.9\textheight,keepaspectratio]{figures/Processchema.pdf}
  \caption{Processchema över systemets helhet.}%
  \label{fig:system_diagram}
\end{figure}

  \subsection{Delsystem A: Bana}
  
  Delsystem A är indelat i tre övergripande delar. I del A.1 hämtas all
  tillgänglig information in, i del A.2a görs beräkningar utifrån tillgänglig
  data, i del A.2b görs vidare beräkningar (alltså beräkningar som inte baseras
  direkt på den tillgängliga informationen), och i del A.3 utförs de ändringar
  som programmet bedömer är nödvändiga för att klara den valda varvtiden och gemensam målgång. 

    \subsubsection{Inhämtning av information}

    Informationen som finns tillgänglig är kraftigt begränsad. I praktiken kommer
    programmet endast fråga om någon av bilarna passerat en givare sedan
    programmet frågade förra gången.

    \subsubsection{Behandling av insignaler}

    De beräkningar som beror direkt på tillgänglig
    information. Då ny indata endast kommer då en bil passerar en givare görs dessa beräkningar inte varje cykel. 
 Ny indata används för att bestämma bilens position och för att kalibrera en konstant. Dessa funktioner beskrivs 
mer ingående i \ref{sec:system_a_funcs}.

    \subsubsection{Vidare beräkningar}
    
    Den första beräkningen som görs är bilens nuvarande position. Detta görs med
    hjälp av en intern bild av banan och vetskapen om vilken hastighet och position bilen
    tidigare haft. Sedan beräknas den position som bäst gör att bilen klarar den satta
    varvtiden ut med hjälp av den nuvarande tiden och, om gemensam målgång är aktiverat, positionen av den andra bilen.  
    I början av varvet görs inte lika drastiska hastighetsändringar som mot slutet.

    Det sista som händer är att informationen om bilens och banans skick används
    för att räkna ut vilket spänningspådrag som krävs för att få bilen att nå
    den hastighet och position som krävs.

    \subsubsection{Utförande}

    I utförandet skickas det nya spänningspådraget till banorna. 
	
    \subsubsection{Funktioner i delsystem A} \label{sec:system_a_funcs}
    I figur~\ref{fig:flow_diagram} visas flödet av de funktioner som sker i delsystem A under en programcykel (alltså 10 gånger per sekund).

    Här listas namn på funktionerna och deras funktion:
    \begin{itemize}
	\item old\_v: Bilens hastighet från olika segment, nuvarande varvet och tidigare lopp. Från denna databas kan andra funktioner få information om hastigheten bilen tidigare haft.
  \item old\_position: Bilens tidigare placering. Från denna databas kan andra funktioner få information om var bilen var förra cykeln, var bilen var för ett varv sedan och så vidare.
      \item indata: Information om huruvida en givare har passerats sedan förra cykeln.

      \item car\_constant: Påverkar new\_u så att new\_u tillsammans med track\_u\_constant motsvarar den hastighet som anges av new\_v. car\_constant ändras endast vid ny indata, vilket innebär att den är konstant under resterande cykler fram tills nästa givare passeras. Genom att jämföra positionen som fås av givarna med indatan kan programmet räkna ut felmarginalen som har uppstått och kalibrera car\_constant new\_u kan justeras med större precision.
 
      \item position: Var på banan bilen befinner sig. Fås genom att hämta senaste positionen (old\_ position) och addera sträckan bilen har färdats sedan dess senaste värde. Sträckan som bilen har färdats kan räknas ut genom $S = v \cdot \delta t$ där $V = \textrm{old\_v}$ samt $\delta t = \textrm{tiden sen senaste cykeln}.$. Om det finns ny indata denna cykel är positionen känd och den faktiska positionen används istället.
      \item clock: Hur länge bilen har varit i det nuvarande segmentet och varvet.

      \item car\_position\_diff: Bilarnas position gentemot varandra. Endast aktiv om gemensam målgång aktiverad. Funktionen utgår från respektive bils placering (old\_position) och hastighet (old\_v)
och ger ett värde på placeringsskillnaden för en viss hastighet. Detta används för att sätta bilarnas nya hastighet. Värdet blir stort om skillnaden i placering är stor men justeras också efter hastigeten. Detta betyder att om bilarna befinner sig långt ifrån varandra men har en hög hastighet blir värdet inte lika stort som om bilarna befinner sig lika långt ifrån varandra men har en lägre hastighet. Värdet är positivt om bilen på bana 1 ligger före bilen på bana 2 och negativt om bilen på bana 2 ligger före bilen på bana 1.
Värdet används sedan för att beräkna nästa hastighet (new\_v) som kommer ökas eller minskas för att få bilarna att köra ikapp varandra. 

      \item target: Sökt varvtid. Sätts manuellt innan programmet startar.
      \item target\_diff: Differensen mellan den önskade tiden och positionen relativt till den faktiska tiden och positionen. Fås genom att subtrahera de önskade värdena med de faktiska värdena. 
 
      \item agressiveness: Justerar hur stora ändringar som görs på new\_v. Vid början av ett varv finns det mycket tid kvar och new\_v kan ändras lite i taget istället för att göra stora förändringar direkt. Det är även onödigt att göra stora ändringar om bilarna befinner sig ungefär där de bör vara. agressiveness räknas ut via clock, hur mycket av varvtiden som återstår, target\_diff och hur långt ifrån målet bilen befinner sig. Om gemensam målgång är aktiv tas även hänsyn till car\_position\_diff.

      \item u\_constant\_map: En kartläggning över banan och de spänningsnivåer som behöver sättas så att spänningen blir jämn. Behövs eftersom spänningstillförseln beter sig olika vid olika delar av banan. Kartläggningen bygger på det register med inlagrad data som tas fram genom tester.
      \item target\_diff: Bilens position relativt till var den borde befinna sig vid den nuvarande tiden.
      
\item track\_u\_ constant: Det förbestämda spänningsvärdet för ett visst subsegment på banan. Tas fram manuellt genom prövning och lagras i u\_constant\_map. Från position tar track\_u\_constant fram rätt spänningsvärde.
     
 \item speed\_map: En kartläggning över banan och hur över hur fort man kan köra i olika delar av banan. Kartläggningen bygger på det register med inlagrad data som tas fram genom tester.

      \item speed\_max: Den förbestämda maxhastigheten för nuvarande subsegment. Tas fram manuellt genom prövning och lagras i speed\_map. Från position tar speed\_constant fram rätt hastighet. 

\item new\_v: Den hastighet som bilen ska få nästa cykel. Tar förra cykelns hastighet (old\_ v) 
och lägger till eller drar av beroende på hur långt ifrån målet bilarna ligger (target\_diff) och, om gemensam
målgång är aktiverad, hur långt ifrån varandra bilarna är (car\_position\_diff). Beror 
också på agressiveness; högre agressiveness ger större skillnad mellan new\_v och old\_v medan ett lågt värde gör att new\_v 
inte ändras särskilt mycket.
new\_v används sedan för att sätta
new\_u. Högre new\_v ger högre new\_u och lägre new\_v ger lägre\_u. 
	
\item new\_u: Den spänning som ska appliceras beroende på vilken hastighet new\_v anger. Ett högre new\_v innebär ett högre new\_u. De andra parametrarna som påverkar new\_u är car\_constant och track\_u\_constant, desto högre dessa värden dessa antar desto högre värde antar också new\_u. new\_u är programmets sista output, dess värde 0 till 127 är det gaspådrag som appliceras på bilen.
    \end{itemize}

    \begin{figure}
      \centering
      \includegraphics[width=\linewidth]{figures/flow.pdf}
      \caption{Funktionsflödet i delsystem A.}%
      \label{fig:flow_diagram}
    \end{figure}

  \subsection{Delsystem B: Display}

  Displayen ter sig enklare än delsystem A. Under körning ska, om ett nytt varv
  påbörjats, den senaste varvtiden och varvnumret skickas till displayen. Om
  stopp-knappen har tryckts ned ska systemet hoppa till resultat-skärmen och om
  inte så ska det fortsätta.

