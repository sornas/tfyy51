\section{Bakgrund}


\section{Syfte och mål}
Syftet med projektet var att konstruera ett system som kör bilar runt en bilbana
%(Se figur~\ref{fig:track_modell}). 
Till bilbanan finns det 9 ``givare'' som när
de passeras skickar en signal till en dator. Genom att mäta tidsskillnaden
mellan signalerna kan man räkna ut hur lång tid det tog för en bil att åka
mellan två givare. Bilbanan är även kopplad till en dator där det finns
möjlighet att justera bilarnas gaspådrag med en spänningstillförsel. Med hjälp
av denna information ska ett system skapas som kör en eller två bilar runt
bilbanan på en inställbar varvtid mellan 12 och 15 sekunder, samt gör att
bilarna åker i mål så nära varandra i tiden som möjligt.
