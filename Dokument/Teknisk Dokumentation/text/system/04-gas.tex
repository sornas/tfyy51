\subsection{Gaspådrag}

Sedan beräknas det gaspådrag som skall sättas till banan. Detta görs i två
funktioner, \emph{get\_new\_v}) och \emph{get\_new\_u}.
 
I \emph{get\_new\_v} används bilens nuvarande postition (\emph{car.postition})
och hastihetskartan (\emph{car.map}). I \emph{car.map} finns en
hastighetsparameter för varje \emph{car.position}, denna retuneras av funktionen
och sparas i \emph{car.v}.
 
I \emph{get\_new\_u} används denna hastighetsparameter tillsammans med
\emph{car.constant}. Dessa multipliceras och deras produkt retuneras och sparas
i \emph{car.u}.

\subsubsection{Governor}

Sedan, om bootstrap är avslutad, körs den del av koden vars ända uppgift är att 
anpassa \emph{car.constant}. 

Detta görs med funktionen \emph{do\_gov}.  Först görs en uppskattning av varvtiden utifrån hur lång tid varvet har tagit än
så länge. Om bilen är inne på sitt första varv görs uppskattningen endast
utifrån förra segmentet \emph{car.forcasts\_naive} och om första varvet är
avslutat använder den i stället \emph{car.forcasts} som kollar på hela varvtiden
fram till och med nu. Detta görs efter segment 4 och 8. Desutom används den
faktiska varvtiden när bilen passerar mål (från varv 2 och frammåt).
 
Sedan jämförs den uppskattade varvtiden med referenstiden \emph{car.ref\_time}.
Om den uppskattade varviden är högre än referenstiden höjs \emph{car.constant}
och om den är lägre sänks \emph{car.constant}.
