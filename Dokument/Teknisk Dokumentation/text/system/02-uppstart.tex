\subsection{Uppstart} 

Vid automatisk körning körs funktionen \emph{do\_boot} vars syfte är att få fram
en initierande konstant (\emph{car\_constant}) och spänningspådrag för den bil
som står på banan. Då bilen är positionerad framför målbågen höjer funktionen
konstanten kontinuerligt i ett tidsintervall på 0.7 sekunder. När väl konstanten
är tillräckligt stor för att bilen ska kunna rulla och passera målbågen så
dämpas höjningen av konstanten och förändringen sker med en lägre frekvens. Vid
passering av den andra givaren så slutar funktionen tillfälligt att förändra
konstanten och låter bilen, med den tilldelade konstanten, åka igenom det tredje
segmentet för att få en uträknad tid. Med tiden det tagit för bilen att ta sig
igenom segmentet räknar funktionen ut vilken förväntad varvtid bilen skulle få
med just den konstanten den hade i segmentet. (beskriva forecastsuträkningen?)
Det sista funktionen gör är att återigen justera konstanten. Om den förväntade
varvtiden är större än 15 sekunder, som är referensvarvtiden för första varvet,
så ökar konstanten och är den förväntade varvtiden mindre än 15 sekunder så
sänks konstanten. 
