\subsection{Uppstart} 
\label{sec:systembeskrivning:uppstart}
Vid autonom körning utgår systemet ifrån en bootstrap som är till för uppstarten av bilarna. Då körs funktionen \texttt{do\_boot()} som arbetar fram en
initial \texttt{car.constant}. Detta sker i tre steg. Innan bilen börjar rulla
höjs \texttt{car.constant} varje 0,7 sekunder. När bilen börjar rulla och åker
under målgivaren höjs \texttt{car.constant} långsammare tills bilen åkt under
den första givaren varpå \texttt{car.constant} inte längre ändras. Vid den
tredje givaren jämförs hur lång tid det senaste segmentet tog att köra och en
sista \texttt{car.constant} räknas ut som förväntas ge en varvtid på 15
sekunder. Om den förväntade varvtiden är längre än 15 sekunder höjs
\texttt{car.constant} och om den förväntade varvtiden är lägre sänks
\texttt{car.constant}.
