\subsection{Körning}
\label{sec:systembeskrivning:korning}

Huvudloopen körs åtminstonde 10 gånger i sekunden. Den beräknar var bilen
befinner sig, hur snabbt bilen ska köra, justerar \texttt{car.constant} och sätter det
nya gaspådraget till banan.

Majoriteten av förändringarna i structen \texttt{car} görs i funktionen \texttt{do\_car()}. 
I \texttt{do\_car()} beräknas  först positionen (\texttt{car.position}), 
sedan motsvarande hastighetsparameter (\texttt{car.v}) och därefter det gaspådrag 
(\texttt{car.u}) som ska sättas till banan.

\subsubsection{Position}
\label{sec:system:korning:position}
Det finns två fall när positionen ska beräknas. När en givare har passerats sedan förra positionsberäkningen och när ingen givare har passerats. Under första varvet hanteras endast det första fallet
och från varv 2 och frammåt görs båda parallellt. 

Om en ny givare har passerats, \texttt{car.new\_check\_point == true}, ökar
programmet nuvarande segment (\texttt{car.segment}) med 1. \texttt{car.segment}, som
alltid ligger mellan 1 och 9, används som index för att välja position i en
lista (\texttt{car.pos\_at}). Vi kallar den positionen för \emph{givarpositionen}.

Om ingen givare har passerars och första varvet är avslutat kallas först på funktionen \texttt{get\_approx\_v()}. Denna utgår ifrån
förra varvets segmentstider (\texttt{car.seg\_times}) och segmentslängder
(\texttt{car.seg\_len}) och beräknar med $v = \frac{s}{t}$, där \texttt{s} är segmentslängden och \texttt{t} segmentstiden, \texttt{v} som är medelhastigheten för nuvarnade
segment, men förra varvet. Denna antas vara ungefär samma som nuvarande
hastiget och kallas \emph{car.v}. 

Sedan beräknas positionen, i meter från målgivaren, med funktionen
\texttt{get\_position()}. Den använder den ungefärliga hastigheten \texttt{v} beräknad av
\texttt{approx\_v()} och tiden \texttt{t} sedan denna beräkning gjordes senast (en programcykel, se \ref{sec:system:korning:cykel})
och beräknar med $s = v \cdot t$ den sträcka som bilen har åkt. Sedan adderas denna
med förra kända positionen och returneras i \texttt{car.position}. Denna 
\emph{beräknade} position tas också fram när en givare har passerats, då skrivs den över med \emph{givarpositionen} men används i stället för att detektera missade givare. Se \ref{sec:missade givare}.

\subsubsection{Gaspådrag}

Efter positionsberäkningen beräknas det gaspådrag som skall sättas till banan. Detta görs i två
funktioner, \texttt{get\_new\_v} och \texttt{get\_new\_u}.
 
I \texttt{get\_new\_v} används bilens nuvarande postition (\texttt{car.postition})
och hastighetskartan (\texttt{car.map}). I \texttt{car.map} finns en
hastighetsparameter för varje \texttt{car.position}(Se \ref{sec:system}.), denna retuneras av funktionen
och sparas i \texttt{car.v}.
 
I \texttt{get\_new\_u} används denna hastighetsparameter tillsammans med
\texttt{car.constant}. Dessa multipliceras och deras produkt retuneras och sparas
i \texttt{car.u}.

\subsubsection{Governor}
\label{sec:systembeskrivning:governor}
Om bootstrap är avslutad, körs den del av koden vars ända uppgift är att 
anpassa \texttt{car.constant}. 

Detta görs med funktionen \texttt{do\_gov}.  Först görs en uppskattning av 
varvtiden utifrån hur lång tid varvet har tagit än
så länge. Detta görs med forecasts som beräknar den approximerade varvtiden utifrån tid fram tills senast
passerad givare samt hastighet i tidigare segment. Genom att veta en
genomsnittlig hastighet går det med kvarvarande sträcka att räkna ut en
ungefärlig kvarvarande tid. När tiden från mål till senaste passerade givare adderas med
den uträknade approximerade tiden kvar, så erhålls det en uppskattad varvtid som
används för att avgöra om en bil behöver åka snabbare eller långsammare. Om bilen dock är inne på sitt första varv görs uppskattningen endast
utifrån förra segmentet \texttt{car.forcasts\_naive} och om första varvet är
avslutat så används \texttt{car.forcasts} som vanligt. Detta görs efter segment 4 och 8. Dessutom används den
faktiska varvtiden när bilen passerar mål (från varv 2 och frammåt).
 
Sedan jämförs denna uppskattade varvtid med referenstiden (\texttt{car.ref\_time}) 
och \texttt{car.constant} justeras.
\begin{verbatim}
car.constant = car.constant + (status - 1) * 0.08;
\end{verbatim}
Där \texttt{status} är den uppskattade varvtidens förhållande till \texttt{car.ref\_time}.
D.v.s om de är exakt lika blir \texttt{status~ =~ 1}, om uppskattningen är högre blir
den större än 1 och om den är lägre blir den mindre än 1. Således kommer \texttt{car.constant}
höjas eller sänkas proportionellt mot hur långt ifrån \texttt{car.ref\_time} uppskattningen
av varvtiden ligger. 

\subsubsection{Hantering av cykellängd}
\label{sec:system:korning:cykel}
I slutet av varje cykel körs det en loop som tillfälligt pausar programmet.
För att få avläsningen att ske minst en gång var tionde sekund så pausas
programmet kontinuerligt 0.001 sekunder tills den totala paustiden överskrider 
0.07 sekunder då nästa cykel börjar. Då pausen på 0.001 sekunder är så pass
kort och marginalen till kravet är rätt stor så sker avläsningen mellan
0.07 och 0.1 sekunder. I en if-sats sparas den högst
uppmätta tiden i \texttt{highToc}.