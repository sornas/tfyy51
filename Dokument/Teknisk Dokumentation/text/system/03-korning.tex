\subsection{Körning}
\label{sec:systembeskrivning:korning}

Körningen är uppdelad i olika faser. Dessa presenteras nedan. En
\emph{programcykel} är när dessa faser körs igenom en gång. Samlingsnamnet för
faserna är \emph{huvudloopen}. Systemet behöver köra programcykler med som högst
0,1~sekunders intervall (enligt krav 31). I en programcykel beräknar systemet
var bilen befinner sig, hur snabbt bilen ska köra och skickar det nya
gaspådraget till banan. När en givare passeras justeras också bilens konstant
efter nuvarande varv- och referenstid.  Nedan beskrivs vad systemet gör under en
komplett programcykel.

\subsubsection{Position}
\label{sec:system:korning:position}

Om en givare passerats vet systemet exakt var bilen befinner sig. Om så är
fallet sparas den nuvarande exakta position och kallas \emph{givarpositionen}.

% Om en ny givare har passerats, \texttt{car.new\_check\_point == true}, ökar
% programmet nuvarande segment (\texttt{car.segment}) med 1.
% \texttt{car.segment}, som alltid ligger mellan 1 och 9, används som index för
% att välja position i en lista (\texttt{car.pos\_at}). Vi kallar den positionen
% för \emph{givarpositionen}.

Oavsett om en givare passerats eller inte räknar systemet ut en beräknad
position. Detta görs genom att räkna ut en approximativ hastighet $v =
\frac{s}{t}$ där $v$ är hastigheten, $s$ är längden på det nuvarande segmentet
och $t$ är hur lång tid det nuvarande segmentet tog förra varvet.

% Om ingen givare har passerars och första varvet är avslutat kallas först på
% funktionen \texttt{get\_approx\_v()}. Denna utgår ifrån förra varvets
% segmentstider (\texttt{car.seg\_times}) och segmentslängder
% (\texttt{car.seg\_len}) och beräknar med $v = \frac{s}{t}$, där \texttt{s} är
% segmentslängden och \texttt{t} segmentstiden, \texttt{v} som är
% medelhastigheten för nuvarnade segment, men förra varvet. Denna antas vara
% ungefär samma som nuvarande hastiget och kallas \emph{car.v}. 

Med den approximativa hastigheten beräknar sedan systemet hur långt bilen rört
sig sedan den senaste programcykeln enligt $\mathrm{d}s = v \cdot \mathrm{d}t$
där $\mathrm{d}s$ är den beräknade förflyttningen, $v$ hastigheten som precis
räknades ut och $\mathrm{d}t$ tiden sedan den förra programcykeln. Denna
förflyttning läggs till positionen från förra programcykeln och kallas den
\emph{beräknade} positionen.

Om en givare passerats uppskattar systemet sedan om en givare vid ett tidigare
tillfälle missats (genom att jämföra givarpositionen och den beräknade
positionen). Information om denna procedur finns i del~\ref{sec:missade givare}.

% Sedan beräknas positionen, i meter från målgivaren, med funktionen
% \texttt{get\_position()}. Den använder den ungefärliga hastigheten \texttt{v}
% beräknad av \texttt{approx\_v()} och tiden \texttt{t} sedan denna beräkning
% gjordes senast (en programcykel, se \ref{sec:system:korning:cykel}) och
% beräknar med $s = v \cdot t$ den sträcka som bilen har åkt. Sedan adderas
% denna med förra kända positionen och returneras i \texttt{car.position}. Denna
% \emph{beräknade} position tas också fram när en givare har passerats, då
% skrivs den över med \emph{givarpositionen} men används i stället för att
% detektera missade givare. Se \ref{sec:missade givare}.

\subsubsection{Gaspådrag}

Efter positionsberäkningen beräknas spänningen som ska skickas till banan. Denna
fås genom att multiplicera en hastighetsparameter ($v$, som beror på bilens
position) med bilens konstant ($k$) enligt $u = kv$.

% Efter positionsberäkningen beräknas spänningen som ska skickas till banan.
% Först används positionen för att välja en spänningsparameter. Denna finns i
% kartan som är indexerad efter position. Gaspådraget ($u$) fås sedan genom att
% multiplicera denna hastighetsparameter ($v$) med bilens konstant ($k$): $u = v
% \cdot k$.

% Efter positionsberäkningen beräknas det gaspådrag som skall sättas till banan.
% Detta görs i två funktioner, \texttt{get\_new\_v} och \texttt{get\_new\_u}.
% 
% I \texttt{get\_new\_v} används bilens nuvarande postition
% (\texttt{car.postition}) och hastighetskartan (\texttt{car.map}). I
% \texttt{car.map} finns en hastighetsparameter för varje \texttt{car.position}
% (Se \ref{sec:begrepp och systemöversikt}.), denna retuneras av funktionen och
% sparas i \texttt{car.v}.
%  
% I \texttt{get\_new\_u} används denna hastighetsparameter tillsammans med
% \texttt{car.constant}. Dessa multipliceras och deras produkt retuneras och
% sparas i \texttt{car.u}.

\subsubsection{Governor}
\label{sec:systembeskrivning:governor}

Om bootstrap är avslutad och en givare passerats behöver bilens konstant
eventuellt justeras. För att justera bilens konstant behöver systemet först
beräkna hur lång tid det nuvarande varvet kommer ta. Vid varje givare beräknas
därför den genomsnittliga hastigheten genom det förra segmentet och
multipliceras med den kvarvarande sträckan på varvet för att få den uppskattade
tiden den resterande delen av varvet kommer ta. Denna tid läggs ihop med
varvtiden hittills och ger då en uppskattad varvtid för det nuvarande varvet.
Denna varvtid jämförs sedan med den valda referenstiden och bilens konstant
anpassas vid behov.

% Detta görs med funktionen \texttt{do\_gov}.  Först görs en uppskattning av
% varvtiden utifrån hur lång tid varvet har tagit än så länge. Detta görs med
% forecasts som beräknar den approximerade varvtiden utifrån tid fram tills
% senast passerad givare samt hastighet i tidigare segment. Genom att veta en
% genomsnittlig hastighet går det med kvarvarande sträcka att räkna ut en
% ungefärlig kvarvarande tid. När tiden från mål till senaste passerade givare
% adderas med den uträknade approximerade tiden kvar, så erhålls det en
% uppskattad varvtid som används för att avgöra om en bil behöver åka snabbare
% eller långsammare. Om bilen dock är inne på sitt första varv görs
% uppskattningen endast utifrån förra segmentet \texttt{car.forcasts\_naive} och
% om första varvet är avslutat så används \texttt{car.forcasts} som vanligt.
% Detta görs efter segment 4 och 8. Dessutom används den faktiska varvtiden när
% bilen passerar mål (från varv 2 och frammåt).
%  
% Sedan jämförs denna uppskattade varvtid med referenstiden
% (\texttt{car.ref\_time}) och \texttt{car.constant} justeras.

% \begin{verbatim}
% car.constant = car.constant + (status - 1) * 0.08;
% \end{verbatim}

% Där \texttt{status} är den uppskattade varvtidens förhållande till
% \texttt{car.ref\_time}.  D.v.s om de är exakt lika blir \texttt{status~ =~ 1},
% om uppskattningen är högre blir den större än 1 och om den är lägre blir den
% mindre än 1. Således kommer \texttt{car.constant} höjas eller sänkas
% proportionellt mot hur långt ifrån \texttt{car.ref\_time} uppskattningen av
% varvtiden ligger. 

\subsubsection{Hantering av längden av en programcykel}
\label{sec:system:korning:cykel}

I slutet av varje cykel körs en loop som tillfälligt pausar programmet. För att
få avläsningen att ske minst en gång var tionde sekund pausas programmet
kontinuerligt i 0,001 sekunder tills den totala paustiden överskrider 0,07
sekunder och nästa cykel börjar. Eftersom pausen på 0,001 sekunder är så pass
kort och marginalen till kravet är stor garanteras att sker med ett intervall på
mellan 0,07 och 0,1 sekunder. Längden på den största skillnaden mellan två
avläsningar sparas och visas vid programmets slut på styrdatorn.

