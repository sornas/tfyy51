\section{Kravbeskrivning}

Krav 1. Systemet är helt skrivet i matlab.

Krav 2. Systemet kan startas oavsett bil på banan.. 

Krav 3. Systemet klarar av att missa givare. 

Krav 4. När ett varv har körts så uppdaterar displayen vilket varv som nyss
genomfördes samt varvtiden. 

Krav 5. Under programmets gång visas det nuvarande gaspådraget. 

Krav 6. Efter att programmet avslutats visas information på displyen.

Krav 7. Systemet kan köras oavsett vilken bil som placeras på banan. 

Krav 8. Programmet hanterar driftsfall genom att kompensera en större eller
mindre styrsignal. 

Krav 9. Om systemet inte får en nt insignal i form av en passerad givare inom
tio sekunder pausas systemet och användaren får frågan om denne vill fortsätta
eller avsluta. 

Krav 10. Användaren har alternativet att köra en eller båda banorna samt hur
banorna ska köras, autonomnt eller manuellt. Det manuella alternativet uppfyller
inte krav 8 på beslut av beställaren. 

Krav 11. Kravet struket från beslut av beställare. 

Krav 12. Tillsammans med frågan om bana ett eller två ska köras frågar
programmet systemet om banan ska köras manuellt eller autonomt. 

Krav 13. För att starta programmet krävs att man kan öppna matlab och starta
programmet. Därefter kan användaren starta med hjälp av displayen.  
 
Krav 14. När systemet startar frågar programmet användaren vilka banor som skall
köras samt vilken referenstid de ska ha. 

Krav 15. Systemet ställer de frågor till användaren via touch displayen.

Krav 16. Enligt de två givna testerna åkte bilarna inte av banan.

Krav 17. När programmet startas frågar programmet användaren vilken referenstid
som ska strävas efter, detta görs i ett intervall ]12,15[ med justeringar på 0,5
sekunder upp eller ner. 

Krav 18. Enligt de två visade körningarna stannade inte bilarna under något
tillfälle. 

Krav 19. 

Krav 20. De två testkörningarna resulterade i en standardavvikelse på 0,22
respektive 0,24. Kravet är delvis uppnått. 

Krav 21. De två testkörningarna resulterade i att bilarna överskred gränsen på
0.5 ett fåtal gånger, kravet delvsi uppnått. 

Krav 22. Kraven var delvis uppfyllda efter 5 varv. 

Krav 23. Kravet struket av beställaren. 

Krav 24. Resultaten sparades och delades med beställaren via email.  

Krav 25. Efter avslutad körning visas statistik i form av de plottar som önskas
i kravspecifikationen. 

Krav 26. Efter avslutad körning sparas alla data i en fil.  

Krav 27. Längre upp i dokumentet beskrivs hur tidtagningen gick till och hur den
validerades.

Krav 28. 

Krav 29. Deltagande i projektet har angett den tid de jobbat efter varje moment. 

Krav 30. Handledaren har inte bidragit med hjälp i mer än 25h.

Krav 31. Efter att programmet avslutas visas den cykel som tog längst tid, då
den inte passerar 0,1 sekunder. 

Krav 32. Efter två veckor av projektet godkänndes projektplanen. 

Krav 33. Under projektvecka fyra godkändes designspecifikationen av beställaren. 

Krav 34. Under projektvecka fem redovisade projektgruppen kraven 2, 4, 31 samt 25. 

Krav 35. Under projektvecka sju redovisade projektgruppen kraven 3, 5, 10, 17
samt 18. Även de krav som uppfylldes under bp.4a visades. 

Krav 36. Under projektvecka nio redovisade projektgruppen samtliga Lrav som
uppfyllts tidigare samt alla krav i avsnitt 3.2.

Krav 37. Programvaran levererades under projektvecka 10. 

Krav 38. Den tekniska dokumentationen levererades under projektvecka 10. 

Krav 39. Under projektvecka tio hölls en slutleverans där gruppen visade upp
samtliga krav och höll en presentation över vad hur arbetet har sett ut. 

Krav 40. Inför varje beslutspunkt har önskade dokument varit beställaren
tillhandahållna innan 09:00 arbetsdagen innan mötet. 

Krav 41. Projektledaren har delat tidsrapportering samt eventuella
mötesprotokoll vid rätt tid de flesta av projektveckorna, kravet är därför
delvis uppnått.

Krav 42. Alla dokument samt all programvara har samlats i gitlab minst en gång i
veckan sedan projektvecka 2. 

Krav 43. Projektplan, designspecifikation, mötesprotokoll, teknisk
dokumentation, testprotokoll samt efterstudie har gjorts. 

Krav 44. Dokument samt programvaran har bearbetats samt lagrats på
http://gitlab.ida.liu.se/.  

Krav 45. Alla dokument framtagna av projektgruppen har levererats i pdf-format. 

Krav 46. Alla dokument skrivna av projektgruppen är är skrivet på formell
korrekt svenska.

Krav 47. Dokumentationen innehåller, 

Krav 48. Programmet är uppdelat i funktioner. 

Krav 49. Projektgruppen har samtlats på mint ett möte i veckan där alla
medlemmar har närvarat. Handledaren har inte närvarat vilket resulterar i ett
delvis uppnått krav. 
