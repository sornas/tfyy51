\section{Handhavande}
\label{app:handhavande}
Starta Matlab 2015b. Observera att användaren måste använda datorn som finns
inne i bilbanerummet och som är inkopplade till bilbanan. Inne i Matlab ska användaren navigera sig till ''Kod'' mappen (som finns tillgängliga för alla i projektet). Börja med att högerklicka på mappen ''Kod'' och välj alternativet ''Add To Path''. Klicka på ''Select Folders And Subfolders'' som dyker upp när musen pekar på ''Add To Path''. Därefter expandera bilbana mappen följt av yc4 mappen. Öppna sedan main.m och starta systemet genom att klicka på Run i Editorn i Matlab.

Därefter välj vilka banor som ska köras via den externa touch displayen. Justera också referenstiden genom att klicka på plusstecknet eller minustecknet (notera att referenstiden ändras med 0.5 s intervall). Kryssa även i om någon av banorna ska köras manuellt. Starta genom att trycka på knappen nere i det högra hörnet på displayen. Observera att bilarna måste placeras några decimeter innan målgivaren före start.

När programmet ska avslutas klickar användaren i kommentar fönstret i Matlab och
klickar på q om bilen ska stanna direkt. Om användaren istället vill att
bilen/bilarna ska stanna precis innan målgivaren klickar användaren på s
(observera att detta fungerar endast efter varv ett).

När programmet avslutas finns möjligheterna att se varvtiden/varvtiderna,
segmentstiden/segmentstiderna samt att avsluta. Detta väljs utifrån de tre
knapparna längst ner på displayen. Dessa knappar är Varv för att se varvtid,
Segment för att se segmentstider. Samt knappen Avsluta.

Om knappen Varv väljs kommer information såsom ''target'' vilket är vald varvtid.
“mean” som är genomsnittlig varvtid och “stdev” är standardavvikelsen. För att
se varvtiden för den andra banan klicka på knappen uppe i högra hörnet.

Om programmet kraschar öppna main.m. Därefter skriv in
ctrl och enter i avgränsningen som heter ''\%\% END OF RACE'' som finns i slutet av
koden main.m.
