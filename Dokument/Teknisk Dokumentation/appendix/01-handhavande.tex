\section{Handhavande}
Proceduren till handhavande för Matlab:

Starta Matlab 2015b. Observera att användaren måste använda datorn som finns
inne i bilbanerummet och som är inkopplade till bilbanan. Väl inne i Matlab
öppna filvägen för hårdisk X: och hitta sökvägen till yc4\_2019.

Därefter markera och högerklicka på mappen kod och där kommer ett alternativ som
heter Add To Path. Välj sedan Select Folders And Subfolders som dyker upp när
musen pekar på Add To Path. Därefter expandera bilbana mappen följt av yc4
mappen. Öppna sedan main.m och starta systemet genom att klicka på Run i Editorn
i Matlab.

När koden körs i Matlab är dags att starta bilarna detta görs genom att via den
externa touch displayen. Där finns möjligheterna att välja antalet banor som ska
köras samt möjligheten att justera referenstiden. Kryssa även i om någon av
banorna ska köras manuellt. Därefter starta genom att trycka på knappen nere i
det högra hörnet på displayen.

När programmet ska avslutas klickar användaren i kommentar fönstret i Matlab och
klickar på q om bilen ska stanna direkt. Om användaren istället vill att
bilen/bilarna ska stanna precis innan målgivaren klickar användaren på s
(observera att detta fungerar endast efter varv ett).

När programmet avslutas finns möjligheterna att se varvtiden/varvtiderna,
segmentstiden/segmentstiderna samt att avsluta. Detta väljs utifrån de tre
knapparna längst ner på displayen. Dessa knappar är Varv för att se varvtid,
Segment för att se segmentstider. Samt knappen Avsluta.

Om knappen Varv väljs kommer information såsom “target” vilket är vald varvtid.
“Mean” som är genomsnittlig varvtid och “Stdev” är standardavvikelsen. För att
se varvtiden för den andra banan klicka på knappen uppe i högra hörnet.

Om programmet kraschar: Om programmet kraschar öppna main.m. Därefter skriv in
ctrl och enter i avgränsningen som heter "\%\% END OF RACE" som finns i slutet av
koden main.m.
