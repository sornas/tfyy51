\documentclass[10pt,oneside,swedish]{lips}
\usepackage{url}

%\usepackage[square]{natbib}\bibliographystyle{plainnat}\setcitestyle{numbers}
\usepackage[round]{natbib}\bibliographystyle{plainnat}

% Configure the document
\title{Biblioteksuppgift}
\author{Yc4}
\date{28 november 2019}
\version{1.0}

\reviewed{ReviewerName}{2015-xx-xx}
\approved{ApproverName}{2015-xx-xx}

\projecttitle{Biblioteksuppgift i TFYY51}

\groupname{Yc4}
\groupemail{groupmail@liu.se}
\groupwww{http://www.isy.liu.se/tsrt10/group}

\coursecode{TFYY51}
\coursename{Ingenjörsprojektet}

\orderer{Beställare, Linköpings universitet}
\ordererphone{+46 xxxxxx}
\ordereremail{ordere@liu.se}

\customer{Kund, Företag X}
\customerphone{+46 xxxxxx}
\customeremail{customer@companyx.com}

\courseresponsible{Boss Person}
\courseresponsiblephone{+46 xxxxxx}
\courseresponsibleemail{the.boss@liu.se}

\supervisor{Handledare}
\supervisorphone{+46 xxxxxx}
\supervisoremail{super.visor@liu.se}

\smalllogo{logo} % Page header logo, filename
\biglogo{logo} % Front page logo, filename

\cfoot{\thepage}
\begin{document}
\maketitle

\cleardoublepage
\makeprojectid

\begin{center}
  \Large Projektdeltagare
\end{center}
\begin{center}
  \begin{tabular}{|l|l|l|}
    \hline
    \textbf{Namn} & \textbf{Ansvar} & \textbf{E-post}\\
    \hline
    Anna Andersson & kundansvarig (KUN) & Annan111@student.liu.se\\
    \hline
    Beata Bson & dokumentansvarig (DOK) & Beabs222@student.liu.se\\
    \hline
    Cecilia Cson & designansvarig (DES) & Ceccs333@student.liu.se\\
    \hline
    Doris Dson & testansvarig (TEST) & Dords444@student.liu.se\\
    \hline
    Erik Eson & kvalitetssamordnare (QA) & Eries555@student.liu.se\\
    \hline
    Fredrik Fson & implementationsansvarig (IMP) & Frefs666@student.liu.se\\
    \hline
    Greta Gson & Projektledare (PL) & Gregs777@student.liu.se\\
    \hline
  \end{tabular}
\end{center}


\cleardoublepage
\tableofcontents

\cleardoublepage
\section*{Dokumenthistorik}
\begin{tabular}{p{.06\textwidth}|p{.1\textwidth}|p{.45\textwidth}|p{.13\textwidth}|p{.13\textwidth}} 
  \multicolumn{1}{c}{\bfseries Version} & 
  \multicolumn{1}{|c}{\bfseries Datum} & 
  \multicolumn{1}{|c}{\bfseries Utförda förändringar} & 
  \multicolumn{1}{|c}{\bfseries Utförda av} & 
  \multicolumn{1}{|c}{\bfseries Granskad}\\
  \hline
  \hline
  0.1 & 2015-11-01 & Första utkast & Sign1 & Name1   \\
  \hline
  0.2 & 2015-11-03 & Första revision & Sign2 & Name2   \\
  \hline
\end{tabular}

\cleardoublepage
\pagenumbering{arabic}\cfoot{\thepage}

\section{Relevanata patent i Europa och USA}
Detta är en text. \emph{Detta är kursiv text}. \textbf{Detta är text i
  fetstil}.


\subsection{Patent i Europa}
Följande alternativ kan specificeras till dokumentmallen:


\subsubsection{}


\subsection{Patent i USA}
En "canister purge method"  (kapselrengöringsmetod) kan minska antalet komponenter för ett aktivt purgesystem. En aktiv purge operation är genomförd med hjälp av att använda tryckvärde med en insugningstrycksenor (intake pressure sensor). Istället för en tryckvärde uppmätt med en bakre tryckgivare (rear-end pressure sensor) efter att en magnetventil (solenoid) för spolningskontroll (purge) är fullt öppen.\footnote{url{
https://worldwide.espacenet.com/publicationDetails/biblio?DB=EPODOC&II=0&ND=3&adjacent=true&locale=en_EP&FT=D&date=20191121&CC=US&NR=2019353112A1&KC=A1}}


\section{Vetenskapliga artiklar}


\subsection{Ytterligare en rubrik på nivå 2}
\lipsum[10]



\end{document}

%%% Local Variables:
%%% mode: latex
%%% TeX-master: t
%%% End:
